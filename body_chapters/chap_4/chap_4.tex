\chapter{Challenges in Test Carving}
% This is a reference to a label in another chapter. Appendix~\ref{chap:app1}. You can always refer to other
% labels from other chapters with the
% \textbf{\textbackslash ref} command.

\section{Common Failure Modalities in Existing Unit Test Carving Tools}
? Look at literature?

\section{Common Failure Modalities of ExploTest}
Two (big) failure modes:
\begin{itemize}
  \item Non-serializable object
    \begin{itemize}
        \item Existence of code objects, things like: \verb|co_name|
      \item Generators, async or non-async
    \end{itemize}
  \item Unable to create oracle
    \begin{itemize}
      \item Benign: FUTs don't return values, they modify class state
    \end{itemize}
\end{itemize}

% \section{}

% It is good practice to follow the latex prefix guidelines for label names:
% \begin{itemize}
%     \item \textit{ch}:  chapter
%     \item \textit{sec}:  section
%     \item \textit{subsec}:  subsection
%     \item \textit{fig}:  figure
%     \item \textit{tab}:  table
%     \item \textit{eq}:  equation
%     \item \textit{lst}:  code listing
%     \item \textit{itm}:  enumerated list item
%     \item \textit{alg}:  algorithm
%     \item \textit{app}:  appendix subsection
% \end{itemize}

% \section{Summary}
% \lipsum[5]
% \section{Future Work}
% \lipsum[6]
